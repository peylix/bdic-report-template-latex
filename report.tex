\documentclass{article}

\usepackage{amsmath, amsthm, amssymb, amsfonts}
\usepackage{thmtools}
\usepackage{graphicx}
\usepackage{setspace}
\usepackage{geometry}
\usepackage{float}
\usepackage{hyperref}
\usepackage[utf8]{inputenc}
\usepackage[english]{babel}
\usepackage{framed}
\usepackage[dvipsnames]{xcolor}
\usepackage{tcolorbox}
\usepackage{multirow}
\usepackage{tabularx}
\usepackage{longtable}
\usepackage{fancyhdr}


% Other
\usepackage{algorithmic}
\usepackage{array}
\usepackage[caption=false,font=footnotesize]{subfig}
\usepackage{lipsum}
\usepackage{hyperref}



% ------------------------------------------------------------------------------
% Name, Course Code and Date
% ------------------------------------------------------------------------------

\newcommand{\name}{Any Student}
\newcommand{\idnumber}{00000000}
\newcommand{\coursecode}{COMP0000J}
\newcommand{\paperdate}{\today}

% ------------------------------------------------------------------------------



\pagestyle{fancy}
\newlength\FHoffset
\setlength\FHoffset{0.1cm}

\addtolength\headwidth{2\FHoffset}
\fancyhead[L]{\name}
\fancyhead[C]{\coursecode}
\fancyhead[R]{\paperdate}



\colorlet{LightGray}{White!90!Periwinkle}
\colorlet{LightOrange}{Orange!15}
\colorlet{LightGreen}{Green!15}

\newcommand{\HRule}[1]{\rule{\linewidth}{#1}}

\declaretheoremstyle[name=Theorem,]{thmsty}
\declaretheorem[style=thmsty,numberwithin=section]{theorem}
\tcolorboxenvironment{theorem}{colback=LightGray}

\declaretheoremstyle[name=Proposition,]{prosty}
\declaretheorem[style=prosty,numberlike=theorem]{proposition}
\tcolorboxenvironment{proposition}{colback=LightOrange}

\declaretheoremstyle[name=Principle,]{prcpsty}
\declaretheorem[style=prcpsty,numberlike=theorem]{principle}
\tcolorboxenvironment{principle}{colback=LightGreen}

\setstretch{1.2}
\geometry{
    textheight=9in,
    textwidth=6.0in,
    top=1in,
    headheight=12pt,
    headsep=25pt,
    footskip=30pt
}

% ------------------------------------------------------------------------------

\begin{document}

% ------------------------------------------------------------------------------
% Cover Page and ToC
% ------------------------------------------------------------------------------

\title{ \normalsize  \includegraphics[width=0.1\textwidth]{images/UCD_Logo.pdf} \textbf{ }
\includegraphics[width=0.15\textwidth]{images/BJUT_Logo.pdf} \par \textsc{Beijing-Dublin International College}
		\\ [2.0cm]
		\HRule{1.5pt} \\ [0.35cm]
		\LARGE \textbf{\uppercase{\coursecode \ Project Report}}
		\HRule{1.5pt} \\ [0.9cm] \textbf{\LARGE{Project Dalek}} \vspace*{10\baselineskip}
		}
\date{\paperdate}
\author{\textbf{Author} \\ 
    \name \ (\idnumber) \\
}

\maketitle
\newpage

\tableofcontents
\newpage

% ------------------------------------------------------------------------------

\begin{abstract}
    %% Text of the abstract
    This template will help you to write your bibliographic and final reports using \LaTeX{}. You'll find here the examples of text structuring as well as tables, figures, citations and references. For other features of \LaTeX, see tutorials on \href{https://www.overleaf.com/learn}{\textbf{Overleaf}} or use this \href{https://wch.github.io/latexsheet/}{\textbf{cheatsheet}}. To work with this template, download its entire folder (including /bibliography and /figures), and run your \LaTeX{}  editor like \href{http://www.xm1math.net/texmaker/}{\textbf{Texmaker}} or \href{https://www.overleaf.com}{\textbf{Overleaf}}. Then make a plan by changing the document's structure with \textit{section} and \textit{subsection} commands. Finally, delete the \textit{lipsum} fillings and start writing you report. Good luck!
\end{abstract}


\section{Introduction}

    \textbf{Example of citation: \cite{Smith_2013} and \cite{Smith_2012}}. \lipsum[2]

\section{Main part}
    \label{S:1}
    \lipsum[9]

    Examples of lists:
    \begin{itemize}
        \item Bullet point one
        \item Bullet point two
    \end{itemize}

    \begin{enumerate}
        \item Numbered list item one
        \item Numbered list item two
    \end{enumerate}

    \subsection{Subsection 1}

        \textbf{Example of table reference: see Table \ref{tab:example}}.
        \lipsum[4]

        \begin{table}[ht] 
            \centering
            \begin{tabular}{l l l}
                \hline
                \textbf{Treatments} & \textbf{Response 1} & \textbf{Response 2}\\
                \hline
                Treatment 1 & 0.0003262 & 0.562 \\
                Treatment 2 & 0.0015681 & 0.910 \\
                Treatment 3 & 0.0009271 & 0.296 \\
                \hline
            \end{tabular}
            \caption{Table caption}
            \label{tab:example}
        \end{table}

    \subsection{Subsection 2}

        \textbf{Example of figure reference: see Figure \ref{fig:example}}. 
        \lipsum[5]

        \begin{figure}[H]
            \centering\includegraphics[width=0.4\linewidth]{figures/placeholder.jpg}
            \caption{Figure caption}
            \label{fig:example}
        \end{figure}

        \textbf{Example of equation reference: see Equation \eqref{eq:emc}}. 
        \lipsum[6]

        \begin{equation} 
            \label{eq:emc}
            e = mc^2
        \end{equation}

    \subsection{Subsection 3}

        \textbf{Example of reference to Section \ref{S:1}.} 
        \lipsum[7]
        \lipsum[8]



\section{Results (optional)}

    \lipsum[9]

\section{Conclusion}

    \lipsum[10]

%----------------------------------------------------------------------------------------
%	Bibliography
%----------------------------------------------------------------------------------------
\bibliography{bibliography/sample}{}
\bibliographystyle{plain}

\end{document}
